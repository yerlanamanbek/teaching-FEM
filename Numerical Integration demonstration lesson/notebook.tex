
% Default to the notebook output style

    


% Inherit from the specified cell style.




    
\documentclass[11pt]{article}

    
    
    \usepackage[T1]{fontenc}
    % Nicer default font (+ math font) than Computer Modern for most use cases
    \usepackage{mathpazo}

    % Basic figure setup, for now with no caption control since it's done
    % automatically by Pandoc (which extracts ![](path) syntax from Markdown).
    \usepackage{graphicx}
    % We will generate all images so they have a width \maxwidth. This means
    % that they will get their normal width if they fit onto the page, but
    % are scaled down if they would overflow the margins.
    \makeatletter
    \def\maxwidth{\ifdim\Gin@nat@width>\linewidth\linewidth
    \else\Gin@nat@width\fi}
    \makeatother
    \let\Oldincludegraphics\includegraphics
    % Set max figure width to be 80% of text width, for now hardcoded.
    \renewcommand{\includegraphics}[1]{\Oldincludegraphics[width=.8\maxwidth]{#1}}
    % Ensure that by default, figures have no caption (until we provide a
    % proper Figure object with a Caption API and a way to capture that
    % in the conversion process - todo).
    \usepackage{caption}
    \DeclareCaptionLabelFormat{nolabel}{}
    \captionsetup{labelformat=nolabel}

    \usepackage{adjustbox} % Used to constrain images to a maximum size 
    \usepackage{xcolor} % Allow colors to be defined
    \usepackage{enumerate} % Needed for markdown enumerations to work
    \usepackage{geometry} % Used to adjust the document margins
    \usepackage{amsmath} % Equations
    \usepackage{amssymb} % Equations
    \usepackage{textcomp} % defines textquotesingle
    % Hack from http://tex.stackexchange.com/a/47451/13684:
    \AtBeginDocument{%
        \def\PYZsq{\textquotesingle}% Upright quotes in Pygmentized code
    }
    \usepackage{upquote} % Upright quotes for verbatim code
    \usepackage{eurosym} % defines \euro
    \usepackage[mathletters]{ucs} % Extended unicode (utf-8) support
    \usepackage[utf8x]{inputenc} % Allow utf-8 characters in the tex document
    \usepackage{fancyvrb} % verbatim replacement that allows latex
    \usepackage{grffile} % extends the file name processing of package graphics 
                         % to support a larger range 
    % The hyperref package gives us a pdf with properly built
    % internal navigation ('pdf bookmarks' for the table of contents,
    % internal cross-reference links, web links for URLs, etc.)
    \usepackage{hyperref}
    \usepackage{longtable} % longtable support required by pandoc >1.10
    \usepackage{booktabs}  % table support for pandoc > 1.12.2
    \usepackage[inline]{enumitem} % IRkernel/repr support (it uses the enumerate* environment)
    \usepackage[normalem]{ulem} % ulem is needed to support strikethroughs (\sout)
                                % normalem makes italics be italics, not underlines
    

    
    
    % Colors for the hyperref package
    \definecolor{urlcolor}{rgb}{0,.145,.698}
    \definecolor{linkcolor}{rgb}{.71,0.21,0.01}
    \definecolor{citecolor}{rgb}{.12,.54,.11}

    % ANSI colors
    \definecolor{ansi-black}{HTML}{3E424D}
    \definecolor{ansi-black-intense}{HTML}{282C36}
    \definecolor{ansi-red}{HTML}{E75C58}
    \definecolor{ansi-red-intense}{HTML}{B22B31}
    \definecolor{ansi-green}{HTML}{00A250}
    \definecolor{ansi-green-intense}{HTML}{007427}
    \definecolor{ansi-yellow}{HTML}{DDB62B}
    \definecolor{ansi-yellow-intense}{HTML}{B27D12}
    \definecolor{ansi-blue}{HTML}{208FFB}
    \definecolor{ansi-blue-intense}{HTML}{0065CA}
    \definecolor{ansi-magenta}{HTML}{D160C4}
    \definecolor{ansi-magenta-intense}{HTML}{A03196}
    \definecolor{ansi-cyan}{HTML}{60C6C8}
    \definecolor{ansi-cyan-intense}{HTML}{258F8F}
    \definecolor{ansi-white}{HTML}{C5C1B4}
    \definecolor{ansi-white-intense}{HTML}{A1A6B2}

    % commands and environments needed by pandoc snippets
    % extracted from the output of `pandoc -s`
    \providecommand{\tightlist}{%
      \setlength{\itemsep}{0pt}\setlength{\parskip}{0pt}}
    \DefineVerbatimEnvironment{Highlighting}{Verbatim}{commandchars=\\\{\}}
    % Add ',fontsize=\small' for more characters per line
    \newenvironment{Shaded}{}{}
    \newcommand{\KeywordTok}[1]{\textcolor[rgb]{0.00,0.44,0.13}{\textbf{{#1}}}}
    \newcommand{\DataTypeTok}[1]{\textcolor[rgb]{0.56,0.13,0.00}{{#1}}}
    \newcommand{\DecValTok}[1]{\textcolor[rgb]{0.25,0.63,0.44}{{#1}}}
    \newcommand{\BaseNTok}[1]{\textcolor[rgb]{0.25,0.63,0.44}{{#1}}}
    \newcommand{\FloatTok}[1]{\textcolor[rgb]{0.25,0.63,0.44}{{#1}}}
    \newcommand{\CharTok}[1]{\textcolor[rgb]{0.25,0.44,0.63}{{#1}}}
    \newcommand{\StringTok}[1]{\textcolor[rgb]{0.25,0.44,0.63}{{#1}}}
    \newcommand{\CommentTok}[1]{\textcolor[rgb]{0.38,0.63,0.69}{\textit{{#1}}}}
    \newcommand{\OtherTok}[1]{\textcolor[rgb]{0.00,0.44,0.13}{{#1}}}
    \newcommand{\AlertTok}[1]{\textcolor[rgb]{1.00,0.00,0.00}{\textbf{{#1}}}}
    \newcommand{\FunctionTok}[1]{\textcolor[rgb]{0.02,0.16,0.49}{{#1}}}
    \newcommand{\RegionMarkerTok}[1]{{#1}}
    \newcommand{\ErrorTok}[1]{\textcolor[rgb]{1.00,0.00,0.00}{\textbf{{#1}}}}
    \newcommand{\NormalTok}[1]{{#1}}
    
    % Additional commands for more recent versions of Pandoc
    \newcommand{\ConstantTok}[1]{\textcolor[rgb]{0.53,0.00,0.00}{{#1}}}
    \newcommand{\SpecialCharTok}[1]{\textcolor[rgb]{0.25,0.44,0.63}{{#1}}}
    \newcommand{\VerbatimStringTok}[1]{\textcolor[rgb]{0.25,0.44,0.63}{{#1}}}
    \newcommand{\SpecialStringTok}[1]{\textcolor[rgb]{0.73,0.40,0.53}{{#1}}}
    \newcommand{\ImportTok}[1]{{#1}}
    \newcommand{\DocumentationTok}[1]{\textcolor[rgb]{0.73,0.13,0.13}{\textit{{#1}}}}
    \newcommand{\AnnotationTok}[1]{\textcolor[rgb]{0.38,0.63,0.69}{\textbf{\textit{{#1}}}}}
    \newcommand{\CommentVarTok}[1]{\textcolor[rgb]{0.38,0.63,0.69}{\textbf{\textit{{#1}}}}}
    \newcommand{\VariableTok}[1]{\textcolor[rgb]{0.10,0.09,0.49}{{#1}}}
    \newcommand{\ControlFlowTok}[1]{\textcolor[rgb]{0.00,0.44,0.13}{\textbf{{#1}}}}
    \newcommand{\OperatorTok}[1]{\textcolor[rgb]{0.40,0.40,0.40}{{#1}}}
    \newcommand{\BuiltInTok}[1]{{#1}}
    \newcommand{\ExtensionTok}[1]{{#1}}
    \newcommand{\PreprocessorTok}[1]{\textcolor[rgb]{0.74,0.48,0.00}{{#1}}}
    \newcommand{\AttributeTok}[1]{\textcolor[rgb]{0.49,0.56,0.16}{{#1}}}
    \newcommand{\InformationTok}[1]{\textcolor[rgb]{0.38,0.63,0.69}{\textbf{\textit{{#1}}}}}
    \newcommand{\WarningTok}[1]{\textcolor[rgb]{0.38,0.63,0.69}{\textbf{\textit{{#1}}}}}
    
    
    % Define a nice break command that doesn't care if a line doesn't already
    % exist.
    \def\br{\hspace*{\fill} \\* }
    % Math Jax compatability definitions
    \def\gt{>}
    \def\lt{<}
    % Document parameters
    \title{index}
    
    
    

    % Pygments definitions
    
\makeatletter
\def\PY@reset{\let\PY@it=\relax \let\PY@bf=\relax%
    \let\PY@ul=\relax \let\PY@tc=\relax%
    \let\PY@bc=\relax \let\PY@ff=\relax}
\def\PY@tok#1{\csname PY@tok@#1\endcsname}
\def\PY@toks#1+{\ifx\relax#1\empty\else%
    \PY@tok{#1}\expandafter\PY@toks\fi}
\def\PY@do#1{\PY@bc{\PY@tc{\PY@ul{%
    \PY@it{\PY@bf{\PY@ff{#1}}}}}}}
\def\PY#1#2{\PY@reset\PY@toks#1+\relax+\PY@do{#2}}

\expandafter\def\csname PY@tok@w\endcsname{\def\PY@tc##1{\textcolor[rgb]{0.73,0.73,0.73}{##1}}}
\expandafter\def\csname PY@tok@c\endcsname{\let\PY@it=\textit\def\PY@tc##1{\textcolor[rgb]{0.25,0.50,0.50}{##1}}}
\expandafter\def\csname PY@tok@cp\endcsname{\def\PY@tc##1{\textcolor[rgb]{0.74,0.48,0.00}{##1}}}
\expandafter\def\csname PY@tok@k\endcsname{\let\PY@bf=\textbf\def\PY@tc##1{\textcolor[rgb]{0.00,0.50,0.00}{##1}}}
\expandafter\def\csname PY@tok@kp\endcsname{\def\PY@tc##1{\textcolor[rgb]{0.00,0.50,0.00}{##1}}}
\expandafter\def\csname PY@tok@kt\endcsname{\def\PY@tc##1{\textcolor[rgb]{0.69,0.00,0.25}{##1}}}
\expandafter\def\csname PY@tok@o\endcsname{\def\PY@tc##1{\textcolor[rgb]{0.40,0.40,0.40}{##1}}}
\expandafter\def\csname PY@tok@ow\endcsname{\let\PY@bf=\textbf\def\PY@tc##1{\textcolor[rgb]{0.67,0.13,1.00}{##1}}}
\expandafter\def\csname PY@tok@nb\endcsname{\def\PY@tc##1{\textcolor[rgb]{0.00,0.50,0.00}{##1}}}
\expandafter\def\csname PY@tok@nf\endcsname{\def\PY@tc##1{\textcolor[rgb]{0.00,0.00,1.00}{##1}}}
\expandafter\def\csname PY@tok@nc\endcsname{\let\PY@bf=\textbf\def\PY@tc##1{\textcolor[rgb]{0.00,0.00,1.00}{##1}}}
\expandafter\def\csname PY@tok@nn\endcsname{\let\PY@bf=\textbf\def\PY@tc##1{\textcolor[rgb]{0.00,0.00,1.00}{##1}}}
\expandafter\def\csname PY@tok@ne\endcsname{\let\PY@bf=\textbf\def\PY@tc##1{\textcolor[rgb]{0.82,0.25,0.23}{##1}}}
\expandafter\def\csname PY@tok@nv\endcsname{\def\PY@tc##1{\textcolor[rgb]{0.10,0.09,0.49}{##1}}}
\expandafter\def\csname PY@tok@no\endcsname{\def\PY@tc##1{\textcolor[rgb]{0.53,0.00,0.00}{##1}}}
\expandafter\def\csname PY@tok@nl\endcsname{\def\PY@tc##1{\textcolor[rgb]{0.63,0.63,0.00}{##1}}}
\expandafter\def\csname PY@tok@ni\endcsname{\let\PY@bf=\textbf\def\PY@tc##1{\textcolor[rgb]{0.60,0.60,0.60}{##1}}}
\expandafter\def\csname PY@tok@na\endcsname{\def\PY@tc##1{\textcolor[rgb]{0.49,0.56,0.16}{##1}}}
\expandafter\def\csname PY@tok@nt\endcsname{\let\PY@bf=\textbf\def\PY@tc##1{\textcolor[rgb]{0.00,0.50,0.00}{##1}}}
\expandafter\def\csname PY@tok@nd\endcsname{\def\PY@tc##1{\textcolor[rgb]{0.67,0.13,1.00}{##1}}}
\expandafter\def\csname PY@tok@s\endcsname{\def\PY@tc##1{\textcolor[rgb]{0.73,0.13,0.13}{##1}}}
\expandafter\def\csname PY@tok@sd\endcsname{\let\PY@it=\textit\def\PY@tc##1{\textcolor[rgb]{0.73,0.13,0.13}{##1}}}
\expandafter\def\csname PY@tok@si\endcsname{\let\PY@bf=\textbf\def\PY@tc##1{\textcolor[rgb]{0.73,0.40,0.53}{##1}}}
\expandafter\def\csname PY@tok@se\endcsname{\let\PY@bf=\textbf\def\PY@tc##1{\textcolor[rgb]{0.73,0.40,0.13}{##1}}}
\expandafter\def\csname PY@tok@sr\endcsname{\def\PY@tc##1{\textcolor[rgb]{0.73,0.40,0.53}{##1}}}
\expandafter\def\csname PY@tok@ss\endcsname{\def\PY@tc##1{\textcolor[rgb]{0.10,0.09,0.49}{##1}}}
\expandafter\def\csname PY@tok@sx\endcsname{\def\PY@tc##1{\textcolor[rgb]{0.00,0.50,0.00}{##1}}}
\expandafter\def\csname PY@tok@m\endcsname{\def\PY@tc##1{\textcolor[rgb]{0.40,0.40,0.40}{##1}}}
\expandafter\def\csname PY@tok@gh\endcsname{\let\PY@bf=\textbf\def\PY@tc##1{\textcolor[rgb]{0.00,0.00,0.50}{##1}}}
\expandafter\def\csname PY@tok@gu\endcsname{\let\PY@bf=\textbf\def\PY@tc##1{\textcolor[rgb]{0.50,0.00,0.50}{##1}}}
\expandafter\def\csname PY@tok@gd\endcsname{\def\PY@tc##1{\textcolor[rgb]{0.63,0.00,0.00}{##1}}}
\expandafter\def\csname PY@tok@gi\endcsname{\def\PY@tc##1{\textcolor[rgb]{0.00,0.63,0.00}{##1}}}
\expandafter\def\csname PY@tok@gr\endcsname{\def\PY@tc##1{\textcolor[rgb]{1.00,0.00,0.00}{##1}}}
\expandafter\def\csname PY@tok@ge\endcsname{\let\PY@it=\textit}
\expandafter\def\csname PY@tok@gs\endcsname{\let\PY@bf=\textbf}
\expandafter\def\csname PY@tok@gp\endcsname{\let\PY@bf=\textbf\def\PY@tc##1{\textcolor[rgb]{0.00,0.00,0.50}{##1}}}
\expandafter\def\csname PY@tok@go\endcsname{\def\PY@tc##1{\textcolor[rgb]{0.53,0.53,0.53}{##1}}}
\expandafter\def\csname PY@tok@gt\endcsname{\def\PY@tc##1{\textcolor[rgb]{0.00,0.27,0.87}{##1}}}
\expandafter\def\csname PY@tok@err\endcsname{\def\PY@bc##1{\setlength{\fboxsep}{0pt}\fcolorbox[rgb]{1.00,0.00,0.00}{1,1,1}{\strut ##1}}}
\expandafter\def\csname PY@tok@kc\endcsname{\let\PY@bf=\textbf\def\PY@tc##1{\textcolor[rgb]{0.00,0.50,0.00}{##1}}}
\expandafter\def\csname PY@tok@kd\endcsname{\let\PY@bf=\textbf\def\PY@tc##1{\textcolor[rgb]{0.00,0.50,0.00}{##1}}}
\expandafter\def\csname PY@tok@kn\endcsname{\let\PY@bf=\textbf\def\PY@tc##1{\textcolor[rgb]{0.00,0.50,0.00}{##1}}}
\expandafter\def\csname PY@tok@kr\endcsname{\let\PY@bf=\textbf\def\PY@tc##1{\textcolor[rgb]{0.00,0.50,0.00}{##1}}}
\expandafter\def\csname PY@tok@bp\endcsname{\def\PY@tc##1{\textcolor[rgb]{0.00,0.50,0.00}{##1}}}
\expandafter\def\csname PY@tok@fm\endcsname{\def\PY@tc##1{\textcolor[rgb]{0.00,0.00,1.00}{##1}}}
\expandafter\def\csname PY@tok@vc\endcsname{\def\PY@tc##1{\textcolor[rgb]{0.10,0.09,0.49}{##1}}}
\expandafter\def\csname PY@tok@vg\endcsname{\def\PY@tc##1{\textcolor[rgb]{0.10,0.09,0.49}{##1}}}
\expandafter\def\csname PY@tok@vi\endcsname{\def\PY@tc##1{\textcolor[rgb]{0.10,0.09,0.49}{##1}}}
\expandafter\def\csname PY@tok@vm\endcsname{\def\PY@tc##1{\textcolor[rgb]{0.10,0.09,0.49}{##1}}}
\expandafter\def\csname PY@tok@sa\endcsname{\def\PY@tc##1{\textcolor[rgb]{0.73,0.13,0.13}{##1}}}
\expandafter\def\csname PY@tok@sb\endcsname{\def\PY@tc##1{\textcolor[rgb]{0.73,0.13,0.13}{##1}}}
\expandafter\def\csname PY@tok@sc\endcsname{\def\PY@tc##1{\textcolor[rgb]{0.73,0.13,0.13}{##1}}}
\expandafter\def\csname PY@tok@dl\endcsname{\def\PY@tc##1{\textcolor[rgb]{0.73,0.13,0.13}{##1}}}
\expandafter\def\csname PY@tok@s2\endcsname{\def\PY@tc##1{\textcolor[rgb]{0.73,0.13,0.13}{##1}}}
\expandafter\def\csname PY@tok@sh\endcsname{\def\PY@tc##1{\textcolor[rgb]{0.73,0.13,0.13}{##1}}}
\expandafter\def\csname PY@tok@s1\endcsname{\def\PY@tc##1{\textcolor[rgb]{0.73,0.13,0.13}{##1}}}
\expandafter\def\csname PY@tok@mb\endcsname{\def\PY@tc##1{\textcolor[rgb]{0.40,0.40,0.40}{##1}}}
\expandafter\def\csname PY@tok@mf\endcsname{\def\PY@tc##1{\textcolor[rgb]{0.40,0.40,0.40}{##1}}}
\expandafter\def\csname PY@tok@mh\endcsname{\def\PY@tc##1{\textcolor[rgb]{0.40,0.40,0.40}{##1}}}
\expandafter\def\csname PY@tok@mi\endcsname{\def\PY@tc##1{\textcolor[rgb]{0.40,0.40,0.40}{##1}}}
\expandafter\def\csname PY@tok@il\endcsname{\def\PY@tc##1{\textcolor[rgb]{0.40,0.40,0.40}{##1}}}
\expandafter\def\csname PY@tok@mo\endcsname{\def\PY@tc##1{\textcolor[rgb]{0.40,0.40,0.40}{##1}}}
\expandafter\def\csname PY@tok@ch\endcsname{\let\PY@it=\textit\def\PY@tc##1{\textcolor[rgb]{0.25,0.50,0.50}{##1}}}
\expandafter\def\csname PY@tok@cm\endcsname{\let\PY@it=\textit\def\PY@tc##1{\textcolor[rgb]{0.25,0.50,0.50}{##1}}}
\expandafter\def\csname PY@tok@cpf\endcsname{\let\PY@it=\textit\def\PY@tc##1{\textcolor[rgb]{0.25,0.50,0.50}{##1}}}
\expandafter\def\csname PY@tok@c1\endcsname{\let\PY@it=\textit\def\PY@tc##1{\textcolor[rgb]{0.25,0.50,0.50}{##1}}}
\expandafter\def\csname PY@tok@cs\endcsname{\let\PY@it=\textit\def\PY@tc##1{\textcolor[rgb]{0.25,0.50,0.50}{##1}}}

\def\PYZbs{\char`\\}
\def\PYZus{\char`\_}
\def\PYZob{\char`\{}
\def\PYZcb{\char`\}}
\def\PYZca{\char`\^}
\def\PYZam{\char`\&}
\def\PYZlt{\char`\<}
\def\PYZgt{\char`\>}
\def\PYZsh{\char`\#}
\def\PYZpc{\char`\%}
\def\PYZdl{\char`\$}
\def\PYZhy{\char`\-}
\def\PYZsq{\char`\'}
\def\PYZdq{\char`\"}
\def\PYZti{\char`\~}
% for compatibility with earlier versions
\def\PYZat{@}
\def\PYZlb{[}
\def\PYZrb{]}
\makeatother


    % Exact colors from NB
    \definecolor{incolor}{rgb}{0.0, 0.0, 0.5}
    \definecolor{outcolor}{rgb}{0.545, 0.0, 0.0}



    
    % Prevent overflowing lines due to hard-to-break entities
    \sloppy 
    % Setup hyperref package
    \hypersetup{
      breaklinks=true,  % so long urls are correctly broken across lines
      colorlinks=true,
      urlcolor=urlcolor,
      linkcolor=linkcolor,
      citecolor=citecolor,
      }
    % Slightly bigger margins than the latex defaults
    
    \geometry{verbose,tmargin=1in,bmargin=1in,lmargin=1in,rmargin=1in}
    
    

    \begin{document}
    
    
    \maketitle
    
    

    
    \subsection{Introduction to Finite Element
Methods}\label{introduction-to-finite-element-methods}

\subsubsection{Gauss quadrature rules}\label{gauss-quadrature-rules}

\subsubsection{Yerlan Amanbek}\label{yerlan-amanbek}

 yerlan@utexas.edu 

    \section{Motivation}\label{motivation}

\begin{itemize}
\tightlist
\item
  Integrant are complicated functions
\item
  Basis function are polynomials of higher order
\end{itemize}

\section{Objective}\label{objective}

\begin{itemize}
\tightlist
\item
  Revisit numerical intergration methods
\item
  Gauss quadrature rule(optimal for polynomials)
\end{itemize}

\subsection{Set-up}\label{set-up}

The notebook relies on Python code. To initialize the notebook, select
\textbf{Cell-\textgreater{}Run All Below}

    \subsection{Introduction}\label{introduction}

We need to compute the definite integral frequently in the FEM
computations.

\[   \int_{a}^b f(x)\,dx \approx \sum_{i=1}^n w_i f(x_i)\]

The integral can be viewed graphically as the area between the
\(x\)-axis and the curve \(y = f (x)\) in the region of the limits of
integration. Thus, we can interpret numerical integration as an
approximation of that area.

    \begin{Verbatim}[commandchars=\\\{\}]
{\color{incolor}In [{\color{incolor}1}]:} \PY{k+kn}{from} \PY{n+nn}{ipywidgets} \PY{k}{import} \PY{n}{interact}\PY{p}{,} \PY{n}{fixed}\PY{p}{,} \PY{n}{IntSlider}\PY{p}{,} \PY{n}{FloatSlider}\PY{p}{,} \PY{n}{Text}\PY{p}{,} \PY{n}{Dropdown}
        \PY{o}{\PYZpc{}}\PY{k}{matplotlib} inline
        \PY{k+kn}{import} \PY{n+nn}{areaTools} \PY{k}{as} \PY{n+nn}{AT}
        \PY{k+kn}{import} \PY{n+nn}{ipywidgets} \PY{k}{as} \PY{n+nn}{widgets}
\end{Verbatim}


    \begin{Verbatim}[commandchars=\\\{\}]
{\color{incolor}In [{\color{incolor}2}]:} \PY{o}{\PYZpc{}}\PY{o}{\PYZpc{}}\PY{n+nx}{javascript}
        \PY{n+nx}{IPython}\PY{p}{.}\PY{n+nx}{OutputArea}\PY{p}{.}\PY{n+nx}{prototype}\PY{p}{.}\PY{n+nx}{\PYZus{}should\PYZus{}scroll} \PY{o}{=} \PY{k+kd}{function}\PY{p}{(}\PY{n+nx}{lines}\PY{p}{)} \PY{p}{\PYZob{}}
            \PY{k}{return} \PY{k+kc}{false}\PY{p}{;}
        \PY{p}{\PYZcb{}}
\end{Verbatim}


    
    \begin{verbatim}
<IPython.core.display.Javascript object>
    \end{verbatim}

    
    \begin{Verbatim}[commandchars=\\\{\}]
{\color{incolor}In [{\color{incolor}3}]:} \PY{c+c1}{\PYZsh{}\PYZsh{} Do not change the text in this box}
        \PY{n}{f\PYZus{}box} \PY{o}{=} \PY{n}{Text}\PY{p}{(}\PY{n}{value}\PY{o}{=}\PY{l+s+s2}{\PYZdq{}}\PY{l+s+s2}{exp(\PYZhy{}x**2)}\PY{l+s+s2}{\PYZdq{}}\PY{p}{,} \PY{n}{description}\PY{o}{=}\PY{l+s+sa}{r}\PY{l+s+s1}{\PYZsq{}}\PY{l+s+s1}{\PYZdl{}f(x)=\PYZdl{}}\PY{l+s+s1}{\PYZsq{}}\PY{p}{)}
        \PY{n}{n\PYZus{}slider} \PY{o}{=} \PY{n}{widgets}\PY{o}{.}\PY{n}{IntSlider}\PY{p}{(}\PY{n+nb}{min}\PY{o}{=}\PY{l+m+mi}{2}\PY{p}{,}\PY{n+nb}{max}\PY{o}{=}\PY{l+m+mi}{20}\PY{p}{,} \PY{n}{step}\PY{o}{=}\PY{l+m+mi}{2}\PY{p}{,} \PY{n}{value}\PY{o}{=}\PY{l+m+mi}{2}\PY{p}{,} \PY{n}{description}\PY{o}{=}\PY{l+s+sa}{r}\PY{l+s+s1}{\PYZsq{}}\PY{l+s+s1}{\PYZdl{}n\PYZus{}}\PY{l+s+si}{\PYZob{}start\PYZcb{}}\PY{l+s+s1}{\PYZdl{}}\PY{l+s+s1}{\PYZsq{}}\PY{p}{)}
        \PY{n}{method\PYZus{}type} \PY{o}{=} \PY{n}{Dropdown}\PY{p}{(}\PY{n}{options}\PY{o}{=}\PY{p}{[}\PY{l+s+s1}{\PYZsq{}}\PY{l+s+s1}{left}\PY{l+s+s1}{\PYZsq{}}\PY{p}{,} \PY{l+s+s1}{\PYZsq{}}\PY{l+s+s1}{right}\PY{l+s+s1}{\PYZsq{}}\PY{p}{,} \PY{l+s+s1}{\PYZsq{}}\PY{l+s+s1}{midpoint}\PY{l+s+s1}{\PYZsq{}}\PY{p}{,}\PY{l+s+s1}{\PYZsq{}}\PY{l+s+s1}{trapezoid}\PY{l+s+s1}{\PYZsq{}}\PY{p}{,}\PY{l+s+s1}{\PYZsq{}}\PY{l+s+s1}{simpson}\PY{l+s+s1}{\PYZsq{}}\PY{p}{]}\PY{p}{,}
                                \PY{n}{value}\PY{o}{=}\PY{l+s+s1}{\PYZsq{}}\PY{l+s+s1}{left}\PY{l+s+s1}{\PYZsq{}}\PY{p}{,} \PY{n}{description}\PY{o}{=}\PY{l+s+s1}{\PYZsq{}}\PY{l+s+s1}{Method:}\PY{l+s+s1}{\PYZsq{}}\PY{p}{)}
        \PY{n}{interact}\PY{p}{(}\PY{n}{AT}\PY{o}{.}\PY{n}{plot3Areas}\PY{p}{,}\PY{n}{f}\PY{o}{=}\PY{n}{f\PYZus{}box}\PY{p}{,}\PY{n}{a}\PY{o}{=}\PY{l+s+s2}{\PYZdq{}}\PY{l+s+s2}{\PYZhy{}1.0}\PY{l+s+s2}{\PYZdq{}}\PY{p}{,}\PY{n}{b}\PY{o}{=}\PY{l+s+s2}{\PYZdq{}}\PY{l+s+s2}{1.0}\PY{l+s+s2}{\PYZdq{}}\PY{p}{,}\PY{n}{n}\PY{o}{=}\PY{n}{n\PYZus{}slider}\PY{p}{,}\PY{n}{method}\PY{o}{=}\PY{n}{method\PYZus{}type}\PY{p}{)}\PY{p}{;}
\end{Verbatim}


    
    \begin{verbatim}
interactive(children=(Text(value='exp(-x**2)', description='$f(x)=$'), Text(value='-1.0', description='a'), Text(value='1.0', description='b'), IntSlider(value=2, description='$n_{start}$', max=20, min=2, step=2), Dropdown(description='Method:', options=('left', 'right', 'midpoint', 'trapezoid', 'simpson'), value='left'), Output()), _dom_classes=('widget-interact',))
    \end{verbatim}

    
    Some quick explanations of how the notebook works:

\textbf{The meaning of the variables are as follows:} - "a": the left
endpoint of the integration interval - "b": the right endpoint of the
integration interval - "n" the number of sub-intervals to use for
approximating the integral - "f(x)": the function to integrate.

\textbf{How to define the function:} - x**p means \(x^p\). For example,
x**0.5 means \(\sqrt x\) - for a constant function \(f(x) = c\) use
\(c*(x**0)\) (every function definition needs an x in it!) - for \(e^p\)
use exp(p) (so for \(e\) use exp(1)) - for \(\pi\) use pi - type trig
functions the way you would on your calculator(sin(x), cos(x), etc)

\textbf{Some functions you might want to try:} - 4-x**2 - cos(x) -
sin(x**2) - exp(x**2) - exp(-x**2/2)/sqrt(2.*pi) (this is the famous
"normal curve") - exp(x)/x (make sure that 0 is not part of your
interval!)

    \[     \int_{a}^b f(x)\,dx \approx \sum_{i=1}^n w_i f(x_i)\]

The trapezoidal rule of numerical integration simply approximates the
area by the sum of several equally spaced trapezoids under the curve
between the limits of a and b.

The height of a trapezoid is found from the integrand,
\(y_j = y ( x_j )\), evaluated at equally spaced points, \(x_j\) and
\(x_{j+1}\). Thus, a typical contribution is
\(Area = h\frac{ y_j + y_{j+1}}{2}\), where \$h = x\_\{j+1\} − x\_j \$
is the spacing. Thus, for \(n\) points (and \(n − 1\) spaces), the
well-known approximation is
\[ I \approx h \left( \frac{1}{2} y_1+y_2+y_3+ ...+y_{n-1}+ \frac{1}{2} y_n \right)\]
where \(w_j = h\), except \(w_1 = w_n = \frac{h}{2}\).

    \begin{Verbatim}[commandchars=\\\{\}]
{\color{incolor}In [{\color{incolor}4}]:} \PY{c+c1}{\PYZsh{}\PYZsh{} Do not change the text in this box}
        \PY{c+c1}{\PYZsh{} \PYZhy{}3*x**2+2*x+6  7*x**3\PYZhy{}8*x**2\PYZhy{}3*x+3}
        \PY{n}{f\PYZus{}box} \PY{o}{=} \PY{n}{Text}\PY{p}{(}\PY{n}{value}\PY{o}{=}\PY{l+s+s2}{\PYZdq{}}\PY{l+s+s2}{\PYZhy{}3*x**2+2*x+6}\PY{l+s+s2}{\PYZdq{}}\PY{p}{,} \PY{n}{description}\PY{o}{=}\PY{l+s+sa}{r}\PY{l+s+s1}{\PYZsq{}}\PY{l+s+s1}{\PYZdl{}f(x)=\PYZdl{}}\PY{l+s+s1}{\PYZsq{}}\PY{p}{)}
        \PY{n}{n\PYZus{}slider} \PY{o}{=} \PY{n}{widgets}\PY{o}{.}\PY{n}{IntSlider}\PY{p}{(}\PY{n+nb}{min}\PY{o}{=}\PY{l+m+mi}{1}\PY{p}{,}\PY{n+nb}{max}\PY{o}{=}\PY{l+m+mi}{20}\PY{p}{,} \PY{n}{step}\PY{o}{=}\PY{l+m+mi}{2}\PY{p}{,} \PY{n}{value}\PY{o}{=}\PY{l+m+mi}{1}\PY{p}{,} \PY{n}{description}\PY{o}{=}\PY{l+s+sa}{r}\PY{l+s+s1}{\PYZsq{}}\PY{l+s+s1}{\PYZdl{}n\PYZus{}}\PY{l+s+si}{\PYZob{}start\PYZcb{}}\PY{l+s+s1}{\PYZdl{}}\PY{l+s+s1}{\PYZsq{}}\PY{p}{)}
        \PY{n}{method\PYZus{}type} \PY{o}{=} \PY{n}{Dropdown}\PY{p}{(}\PY{n}{options}\PY{o}{=}\PY{p}{[}\PY{l+s+s1}{\PYZsq{}}\PY{l+s+s1}{trapezoid}\PY{l+s+s1}{\PYZsq{}}\PY{p}{]}\PY{p}{,}
                                \PY{n}{value}\PY{o}{=}\PY{l+s+s1}{\PYZsq{}}\PY{l+s+s1}{trapezoid}\PY{l+s+s1}{\PYZsq{}}\PY{p}{,} \PY{n}{description}\PY{o}{=}\PY{l+s+s1}{\PYZsq{}}\PY{l+s+s1}{Method:}\PY{l+s+s1}{\PYZsq{}}\PY{p}{)}
        \PY{n}{interact}\PY{p}{(}\PY{n}{AT}\PY{o}{.}\PY{n}{plot3Areas}\PY{p}{,}\PY{n}{f}\PY{o}{=}\PY{n}{f\PYZus{}box}\PY{p}{,}\PY{n}{a}\PY{o}{=}\PY{l+s+s2}{\PYZdq{}}\PY{l+s+s2}{\PYZhy{}1.0}\PY{l+s+s2}{\PYZdq{}}\PY{p}{,}\PY{n}{b}\PY{o}{=}\PY{l+s+s2}{\PYZdq{}}\PY{l+s+s2}{1.0}\PY{l+s+s2}{\PYZdq{}}\PY{p}{,}\PY{n}{n}\PY{o}{=}\PY{n}{n\PYZus{}slider}\PY{p}{,}\PY{n}{method}\PY{o}{=}\PY{n}{method\PYZus{}type}\PY{p}{)}\PY{p}{;}
\end{Verbatim}


    
    \begin{verbatim}
interactive(children=(Text(value='-3*x**2+2*x+6', description='$f(x)=$'), Text(value='-1.0', description='a'), Text(value='1.0', description='b'), IntSlider(value=1, description='$n_{start}$', max=20, min=1, step=2), Dropdown(description='Method:', options=('trapezoid',), value='trapezoid'), Output()), _dom_classes=('widget-interact',))
    \end{verbatim}

    
    \section{Gauss integration}\label{gauss-integration}

The famous mathematician Gauss posed this question:

\begin{itemize}
\item
  What is the minimum number of points, \(n\), required to exactly
  integrate a polynomial?
\item
  What are the corresponding function values and weights?
\end{itemize}

    \section{Example:}\label{example}

Let \([a, b] = [-1, 1]\) and \(n=2\)

We want to choose \(x_1, x_2, w_1\) and \(w_2\) so that

\[     \int_{-1}^1 f(x)\,dx \approx w_1 f(x_1) + w_2 f(x_2)\]

The approximation should be exact for any polynomial of degree 3 or
less.

    Let \(f(x) = a_0 + a_1x+a_2x^2+a_3x^3\)

then

\(\int_{-1}^1 f(x)\,dx = a_0\int_{-1}^1 1\,dx+a_1\int_{-1}^1 x\,dx+a_2\int_{-1}^1 x^2\,dx+a_3\int_{-1}^1 x^3\,dx\)

Each of the definite integrals on the right-hand side has an integrand
of degree 3 or less. Gaussian Quadrature should be exact for each of
these.

    Applying Gaussian Quadrature to each remaining integral yields:

\[\int_{-1}^1 1\,dx = 2 = w_1f(x_1) + w_2f(x_2)=w_1 + w_2\]

    \[\int_{-1}^1 x \,dx = 0 = w_1f(x_1) + w_2f(x_2)= w_1 x_1 +w_2 x_2\]

    \[\int_{-1}^1 x^2\,dx = \frac{2}{3} = w_1 x_1^2+w_2x_2^2\]

    \[ \int_{-1}^1 x^3\,dx = 0 =  w_1x_1^3 +w_2 x_2^3\] We must solve this
system of 4 nonlinear equations in 4 unknowns.

    \(w_1 + w_2 = 2\)

\(w_1 x_1 +w_2 x_2 = 0\) we obtain \(w_1 = \frac{2x_2}{x_2-x_1}\) and
\(w_2 = \frac{2x_1}{x_1-x_2}\)

Using these values for \(3^{rd}\) equation \(x_1 x_2 =-\frac{1}{3}\) and
\(4^{th}\) equation \(x_1 = -x_2\).

    Solving the nonlinear system gives us

\(w_1 =1\), \(w_2=1\), \(x_1 = - \frac{\sqrt 3}{3}\),
\(x_2 = \frac{\sqrt 3}{3}\)

Therefore,

\[\int_{-1}^1 f(x)\,dx \approx f \left( \frac{-\sqrt 3}{3}\right)+ f \left( \frac{\sqrt 3}{3}\right)\]

Above Gauss quadrature integral has degree of precision 3. Trapezoidal
rule has degree of precision 1.

    \section{Gauss Quadrature rule}\label{gauss-quadrature-rule}

\[     \int_{-1}^1 f(x)\,dx \approx \sum_{i=1}^n w_i f(x_i)\]

Reference: Stoer, Josef; Bulirsch, Roland (2002), Introduction to
Numerical Analysis (3rd ed.), Springer, ISBN 978-0-387-95452-3

    \section{Example}\label{example}

\(\int_{-1}^1 \left( x^2+\cos{x} \right)\,dx\) =
\(\frac{x^3}{3} \bigg\rvert_{-1}^{1} +\sin{x}\bigg\rvert_{-1}^{1} =\frac{2}{3}+2\sin(1)\)

Trapezoidal rule :

\(\int_{-1}^1 \left( x^2+\cos{x} \right)\,dx \approx \frac{2}{2}\left( f \left( -1\right)+ f \left( 1\right)\right)\)

Gauss quadrature:

\(\int_{-1}^1 \left( x^2+\cos{x} \right)\,dx \approx f \left( \frac{-\sqrt 3}{3}\right)+ f \left( \frac{\sqrt 3}{3}\right)\)

    \begin{Verbatim}[commandchars=\\\{\}]
{\color{incolor}In [{\color{incolor}5}]:} \PY{k+kn}{from} \PY{n+nn}{scipy}\PY{n+nn}{.}\PY{n+nn}{integrate} \PY{k}{import} \PY{n}{quad}
        \PY{k+kn}{from} \PY{n+nn}{numpy} \PY{k}{import} \PY{n}{sqrt}\PY{p}{,} \PY{n}{sin}\PY{p}{,} \PY{n}{cos}\PY{p}{,} \PY{n}{pi}
        
        \PY{k}{def} \PY{n+nf}{f}\PY{p}{(}\PY{n}{x}\PY{p}{)}\PY{p}{:}
            \PY{k}{return} \PY{n}{x}\PY{o}{*}\PY{o}{*}\PY{l+m+mi}{2} \PY{o}{+} \PY{n}{cos}\PY{p}{(}\PY{n}{x}\PY{p}{)} \PY{c+c1}{\PYZsh{}integrant}
\end{Verbatim}


    \begin{Verbatim}[commandchars=\\\{\}]
{\color{incolor}In [{\color{incolor}6}]:} \PY{n}{I\PYZus{}actual} \PY{o}{=}\PY{l+m+mi}{2}\PY{o}{/}\PY{l+m+mi}{3} \PY{o}{+} \PY{l+m+mi}{2}\PY{o}{*}\PY{n}{sin}\PY{p}{(}\PY{l+m+mi}{1}\PY{p}{)} \PY{c+c1}{\PYZsh{}computed integral}
        
        \PY{n+nb}{print}\PY{p}{(}\PY{l+s+s2}{\PYZdq{}}\PY{l+s+s2}{Analytical computation of integral       is }\PY{l+s+si}{\PYZpc{}1.10f}\PY{l+s+s2}{\PYZdq{}} \PY{o}{\PYZpc{}} \PY{n}{I\PYZus{}actual}\PY{p}{)}
        
        \PY{c+c1}{\PYZsh{}Trapezoidal rule for integral computation}
        \PY{n}{h}\PY{o}{=}\PY{l+m+mi}{2}
        \PY{n}{I\PYZus{}trapezoidal} \PY{o}{=} \PY{n}{h}\PY{o}{*}\PY{p}{(}\PY{n}{f}\PY{p}{(}\PY{o}{\PYZhy{}}\PY{l+m+mi}{1}\PY{p}{)}\PY{o}{+}\PY{n}{f}\PY{p}{(}\PY{l+m+mi}{1}\PY{p}{)}\PY{p}{)}\PY{o}{/}\PY{l+m+mi}{2}
        \PY{n}{error\PYZus{}trapezoidal} \PY{o}{=} \PY{n+nb}{abs}\PY{p}{(}\PY{n}{I\PYZus{}actual} \PY{o}{\PYZhy{}} \PY{n}{I\PYZus{}trapezoidal}\PY{p}{)}                   
        \PY{n+nb}{print}\PY{p}{(}\PY{l+s+s2}{\PYZdq{}}\PY{l+s+s2}{Trapezoidal rule computation of integral is }\PY{l+s+si}{\PYZpc{}1.10f}\PY{l+s+s2}{ and error is }\PY{l+s+si}{\PYZpc{}1.10f}\PY{l+s+s2}{\PYZdq{}} \PY{o}{\PYZpc{}} \PY{p}{(}\PY{n}{I\PYZus{}trapezoidal}\PY{p}{,} \PY{n}{error\PYZus{}trapezoidal}\PY{p}{)}\PY{p}{)}
\end{Verbatim}


    \begin{Verbatim}[commandchars=\\\{\}]
Analytical computation of integral       is 2.3496086363
Trapezoidal rule computation of integral is 3.0806046117 and error is 0.7309959755

    \end{Verbatim}

    \begin{Verbatim}[commandchars=\\\{\}]
{\color{incolor}In [{\color{incolor}7}]:} \PY{c+c1}{\PYZsh{}Gauss quadrature rule for integral computation}
        \PY{n}{I\PYZus{}Gauss\PYZus{}Quadrature} \PY{o}{=} \PY{n}{f}\PY{p}{(}\PY{o}{\PYZhy{}}\PY{n}{sqrt}\PY{p}{(}\PY{l+m+mi}{3}\PY{p}{)}\PY{o}{/}\PY{l+m+mi}{3}\PY{p}{)}\PY{o}{+}\PY{n}{f}\PY{p}{(}\PY{n}{sqrt}\PY{p}{(}\PY{l+m+mi}{3}\PY{p}{)}\PY{o}{/}\PY{l+m+mi}{3}\PY{p}{)}
        \PY{n}{error\PYZus{}gauss} \PY{o}{=} \PY{n+nb}{abs}\PY{p}{(}\PY{n}{I\PYZus{}actual} \PY{o}{\PYZhy{}} \PY{n}{I\PYZus{}Gauss\PYZus{}Quadrature}\PY{p}{)}
        \PY{n+nb}{print}\PY{p}{(}\PY{l+s+s2}{\PYZdq{}}\PY{l+s+s2}{Analytical computation of integral       is }\PY{l+s+si}{\PYZpc{}1.10f}\PY{l+s+s2}{\PYZdq{}} \PY{o}{\PYZpc{}} \PY{n}{I\PYZus{}actual}\PY{p}{)}
        \PY{n+nb}{print}\PY{p}{(}\PY{l+s+s2}{\PYZdq{}}\PY{l+s+s2}{Gauss quadrature computation of integral is }\PY{l+s+si}{\PYZpc{}1.10f}\PY{l+s+s2}{ and error is }\PY{l+s+si}{\PYZpc{}1.10f}\PY{l+s+s2}{\PYZdq{}} \PY{o}{\PYZpc{}} \PY{p}{(}\PY{n}{I\PYZus{}Gauss\PYZus{}Quadrature}\PY{p}{,} \PY{n}{error\PYZus{}gauss}\PY{p}{)} \PY{p}{)}
\end{Verbatim}


    \begin{Verbatim}[commandchars=\\\{\}]
Analytical computation of integral       is 2.3496086363
Gauss quadrature computation of integral is 2.3424903221 and error is 0.0071183142

    \end{Verbatim}

    \section{Gauss quadrature on arbitrary
intervals}\label{gauss-quadrature-on-arbitrary-intervals}

Use substition or transformation to transform \$\int\_\{a\}\^{}b f(x),dx
\$ into an integral defined over {[}-1, 1{]}

Let \(x = \frac{1}{2}(a+b)+\frac{1}{2}(b-a)t\), with \(t \in [-1, 1]\)
Then

\[ \int_{a}^b f(x)\,dx = \int_{-1}^1 f\left(\frac{1}{2}(a+b)+\frac{1}{2}(b-a)t \right) \frac{b-a}{2}\,dt\]

\subsubsection{Remark}\label{remark}

\begin{itemize}
\tightlist
\item
  Using \(n\) quadrature points, a polynomial \(P(x)\) of degree
  \((2n – 1)\) or less will be integrated exactly.
\item
  If computational efforts being equal,Gaussian integration yields the
  most accurate results.
\end{itemize}

    \subsection{Summary}\label{summary}

\begin{itemize}
\item
  Numerical experiments
\item
  Gauss quadrature rule:

  \begin{itemize}
  \tightlist
  \item
    using \(n\) quadrature points, a polynomial P(x) of degree (2n -- 1)
    or less will be integrated exactly.
  \item
    Compared to trapezoidal rule
  \end{itemize}
\end{itemize}

\subsubsection{Homework}\label{homework}

Exercise 2.3, 2.5 and Numerical experiments with functions from 2.6


    % Add a bibliography block to the postdoc
    
    
    
    \end{document}
